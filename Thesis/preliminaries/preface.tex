\chapter*{Preface}
\pdfbookmark{Preface}{Preface}
\chaptermark{Preface}
\sectionmark{Preface}
You will probably want to replace this text with the preface of your document.
In the mean time, the coming section provides a summary of how this LaTeX template is structured.
\begin{description}
\item[master.tex:]
	This is the master document which is compiled by LaTeX. This is where you refer to all the other LaTeX files that are a part of the, and also where you define the metadata such as the author, supervisor, and title of the document.

\item[master.ist:]
	This file contains the style for the index that will appear at the back
	of the document. You probably won't need to touch this file. You can add
	entries to the index by using the macro \texttt{\textbackslash index} in your regular \texttt{.tex}
	files.

\item[library.bib:]
	This is a BibTeX file that contains a database over all the books and
	papers that you intend to cite. Only references that are actually used will
	appear in the bibliography; so remember that anything you add to this file
	will be invisible in your document until you \texttt{\textbackslash cite} it. This bibliography
	database can be written and maintained manually (check the included example,
	or google bibtex to find a guide), but I highly recommend that you instead
	use some automated tool like Mendeley to generate it for you.

\item[preamble/include.tex:]
	This is where you should include new LaTeX packages using the \texttt{\textbackslash usepackage} macro, or perhaps tweak the arguments of existing packages.

\item[preamble/input.tex:]
	This is where you declare unicode symbols that you wish to use in LaTeX.

\item[preamble/style.tex:]
	This is where you declare stylistic properties of the document, such as margins,
	headers, footers, and so on.

\item[preamble/macro.tex:]
	This is where you should define your custom LaTeX macros.

\item[preliminaries/*.tex]
	This is the location of the titlepage, abstract and preface of your document.

\item[chapters/*.tex]
	This folder should contain all the chapters of your document as separate .tex files.
	Remember that for every .tex file you add here, you should also add a corresponding
	statement \texttt{\textbackslash include{mainmatter/filename.tex}} in \texttt{master.tex} for the contents to appear
	in the document.

\item[appendices/*.tex]
	This folder should contain your appendices as separate .tex files. Remember to add
	a corresponding \texttt{\textbackslash include{backmatter/filename.tex}} in \texttt{master.tex}.

\end{description}
